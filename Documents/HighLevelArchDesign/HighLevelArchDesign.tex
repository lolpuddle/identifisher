\documentclass[]{article}

% Imported Packages
%------------------------------------------------------------------------------
\usepackage{amssymb}
\usepackage{amstext}
\usepackage{amsthm}
\usepackage{amsmath}
\usepackage{enumerate}
\usepackage{fancyhdr}
\usepackage[margin=1in]{geometry}
\usepackage{graphicx}
\usepackage{extarrows}
\usepackage{setspace}
%------------------------------------------------------------------------------

% Header and Footer
%------------------------------------------------------------------------------
\pagestyle{plain}  
\renewcommand\headrulewidth{0.4pt}                                      
\renewcommand\footrulewidth{0.4pt}                                    
%------------------------------------------------------------------------------

% Title Details
%------------------------------------------------------------------------------
\title{Deliverable \#2 Template}
\author{SE 3A04: Software Design II -- Large System Design}
\date{}                               
%------------------------------------------------------------------------------

% Document
%------------------------------------------------------------------------------
\begin{document}

\maketitle	

\section{Introduction}
\label{sec:introduction}
% Begin Section

This section should provide a brief overview of the entire document.

\subsection{Purpose}
\label{sub:purpose}
% Begin SubSection
\begin{enumerate}[a)]
	\item
	The purpose of this document is to present the high-level architectural design of the IdentiFisher Android application introduced in the Software Specification Requirements document. This will be accomplished using various diagrams and textual descriptions.
	\item 
	The intended audience for this document includes any stakeholders involved in the project or interested in the application. This document is especially intended for any person on the project development team who will have a role in the design and implementation of the application and may also include investors, managers, or future users of the application who wish to see the high-level design of the application.  
\end{enumerate}
% End SubSection

\subsection{System Description}
\label{sub:system_description}
% Begin SubSection
\begin{enumerate}[]
	\item 
	The system described in this document is called IdentiFisher, an Android application intended to be used by beginner to experienced fishers. IdentiFisher accepts user input about a fish that they have caught and attempts to identify the type based on specific details about the physical appearance and geolocation of the fish. The application will interface with an online mapping system in order to obtain geolocational information and will maintain a data collection of fish caught in specific locations in order to generate catch-rate and other statistics that will be available to users of the application upon request. 
\end{enumerate}
% End SubSection

\subsection{Overview}
\label{sub:overview}
% Begin SubSection
	The beginning of this document has introduced the purpose of the document and has given a brief outline of IdentiFisher, the system being designed. The subsequent sections will go into detail about the uses and high-level design of the application. First of all, a Use Case Diagram will be presented with a description of the uses of the application. Next, an Analysis Class Diagram will be included in order to show the general organization of the application's classes. The fourth section will provide an overview of the architectural design of the application, including a Structural Architecture Diagram as well as a description of any subsystems. The fifth section will be comprised of Class Responsibility Collaboration (CRC) Cards that will outline the responsibilities and collaborators for each class identified. Finally, a Division of Labour section is included in this document in order to identify each author along with the portion so the document that they have completed. 
% End SubSection

% End Section

\section{Use Case Diagram}
\label{sec:use_case_diagram}
% Begin Section
This section should provide a use case diagram for your application. 
\begin{enumerate}[a)]
	\item Each use case appearing in the diagram should be accompanied by a text description. 
\end{enumerate}
% End Section

\section{Analysis Class Diagram}
\label{sec:analysis_class_diagram}
% Begin Section
This section should provide an analysis class diagram for your application.
% End Section


\section{Architectural Design}
\label{sec:architectural_design}
% Begin Section
This section should provide an overview of the overall architectural design of your application. You overall architecture should show the division of the system into subsystems with high cohesion and low coupling.

\subsection{System Architecture}
\label{sub:system_architecture}
% Begin SubSection
\begin{enumerate}[a)]
	\item Identify and explain the overall architecture of your system
	\item Be sure to clearly state the name of the architecture
	\item Provide the reasoning and justification of the choice
	\item Provide a structural architecture diagram showing the relationship among the subsystems (if appropriate)
\end{enumerate}
% End SubSection

\subsection{Subsystems}
\label{sub:subsystems}
% Begin SubSection
\begin{enumerate}[a)]
	\item Provide a brief description of each subsystem. Be sure to document its purpose and relationship to other subsystems.
\end{enumerate}
% End SubSection

% End Section
	
\section{Class Responsibility Collaboration (CRC) Cards}
\label{sec:class_responsibility_collaboration_crc_cards}
% Begin Section
This section should contain all of your CRC cards.

\begin{enumerate}[a)]
	\item Provide a CRC Card for each identified class
	\item Please use the format outlined in tutorial, i.e., 
	\begin{table}[ht]
		\centering
		\begin{tabular}{|p{5cm}|p{5cm}|}
		\hline 
		 \multicolumn{2}{|l|}{\textbf{Class Name:}} \\
		\hline
		\textbf{Responsibility:} & \textbf{Collaborators:} \\
		\hline
		\vspace{1in} & \\
		\hline
		\end{tabular}
	\end{table}
	
\end{enumerate}
% End Section

\appendix
\section{Division of Labour}
\label{sec:division_of_labour}
% Begin Section
\begin{center}
\begin{tabular}{ |c|c|c| } 
 \hline
 Name & Labour & Signature              \\ \hline
 Shani & Introduction I & \\ 
 Chris & Class Responsibility Charts &  \\
 James & Architecture Design &  \\ 
 Ocean & Analysis Class Diagram &  \\ 
 Tian & Use Case Diagram & \\ 
 \hline
\end{tabular}
\end{center}
% End Section

\newpage
\section*{IMPORTANT NOTES}
\begin{itemize}
%	\item You do \underline{NOT} need to provide a text explanation of each diagram; the diagram should speak for itself
	\item Please document any non-standard notations that you may have used
	\begin{itemize}
		\item \emph{Rule of Thumb}: if you feel there is any doubt surrounding the meaning of your notations, document them
	\end{itemize}
	\item Some diagrams may be difficult to fit into one page
	\begin{itemize}
		\item It is OK if the text is small but please ensure that it is readable when printed
		\item If you need to break a diagram onto multiple pages, please adopt a system of doing so and thoroughly explain how it can be reconnected from one page to the next; if you are unsure about this, please ask about it
	\end{itemize}
	\item Please submit the latest version of Deliverable 1 with Deliverable 2
	\begin{itemize}
		\item It does not have to be a freshly printed version; the latest marked version is OK
	\end{itemize}
	\item If you do \underline{NOT} have a Division of Labour sheet, your deliverable will \underline{NOT} be marked
\end{itemize}


\end{document}
%------------------------------------------------------------------------------