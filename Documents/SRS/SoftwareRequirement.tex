\documentclass{article}
\usepackage{graphicx}
\usepackage{float}

\title{Software Specification Requirements: \\ IdentiFisher \\ Group 7}

\author{
McDonald, Christopher\\
\texttt{1312456}
\and
Guo, Tian\\
\texttt{1327833}
\and
Murray, Shandelle\\
\texttt{1303109}
\and
Cheung, Ocean\\
\texttt{000000}
\and
Taylor, James\\
\texttt{000000}
}
\vfill
\date{\today}

\begin{document}
\maketitle

\newpage

\tableofcontents

\vspace*{\fill}
Revision 0: This is the first draft written from the authors listed on the Title page.

\section{Introduction}

\subsection{Purpose}
The purpose of the SRS is to provide a detailed account of all the expected functions
and requirements of the Software system. It will go into detail regarding the system
as a whole, who we expect to use it and any relevant information one would need
to endorse or build the system. Lastly, we will outline both the functional and
non-functional requirements of the project that are necessary for the system's
success. The intended audience of this document is any shareholders that are involved in this
project. This could include, but is not limited to; the investors, developers, managers,
marketers or human resource workers. Every person which is an entity in the aforementioned
list should all take an interest in the details outlined hereafter to ensure every person
has a clear idea of what the software system should do.

\subsection{Scope}
The software system will be named hereafter as IdentiFisher, which is an Android Application.
This system will be a utility application for anyone who fishes, either recreationally or
competivitely. It also will service novice to experienced fishers. Identifisher will allow
the user to give information about a recently caught fish and help to identify what type
of fish it is. From there, it can collect data and track what fish are caught where. We
hope to build a global logging system that will provide percentage catch rates by lake,
educate young, novice fishers and integrate technology into a relatively non-technology field.


\subsection{Definitions, Acronyms, and Abbreviations}

\iffalse
Anything can be added either by anyone on the team including myself, Chris.
\fi

\subsection{References}

\iffalse
Same goes for this, anything can be added as needed. I can cite if you wish.
\fi

\subsection{Overview}
Thus far we have given a very brief overview of the IdentiFisher application, its intended
use and what we expect a typical user would be. Going forward, we will go into deeper detail
regarding those topics and more. The next section will give far more information regarding
the application and some of the external matters regarding the system. After that, functional
requirements will be listed with non-functional requirements being the last section of this document.

\section{Overall Description}

\iffalse
This is still a rough draft of the overall desciption. If anyone would like to edit/add to this section, feel free.
\fi

\subsection{Product Perspective}

The IdentiFisher application is similar to other applications that, by user request, analyze textual input or images in order to identify an entity. It is independent as it is not intended to be used as part of a larger system; however, it will interface with an online mapping system in order to perform geolocational functions.  

\subsection{Product Functions}
The IdentiFisher application will allow the user to input textual or pictorial data representing a specific fish, request an estimation of the type of fish described, and request statistical information about fish in a specific location. Secondly, the application must interface with an online mapping system in order to determine the geolocation of the user. The application must also be able to access a collection of data related to fish population statistics. Finally, the application must determine and display reasonable predictions about the type of fish the user has described. 

\begin{figure}[H]
	\includegraphics[scale=0.7]{images/contextdiagram}
	\caption{Context Diagram of IdentiFisher Application}
\end{figure}


\subsection{User Characteristics}
The IdentiFisher application is intended to be used by novice to experienced fishers who wish to identify captured fish or access geographical fish population statistics. 

Technologically, an intended user of the application should have access to a device on which they are capable of installing and accessing an application, establishing an internet connection, as well as generating and inputting textual and pictorial data. 

\subsection{Constraints}

Since the statistics about a lake are generated by actual users of the application, the method by which data is collected must be subject to integrity constraints. The application must employ a method of data verification before adding it to the statistics available to other users of the application. 

\subsection{Assumptions \& Dependencies}

An important assumption is that a user of the application is inputting data which represents a real fish and not data that is fake or replicated from another source. This includes inputting images from the internet or simply inputting details about a fish that the user has not actually found in a specific lake. 

\subsection{Apportioning of Requirements}

\section{Functional Requirements}

\section{Non-Functional Requirements}

\subsection{Look and Feel Requirements}
The IdentiFisher shall be pleasing to look at
\subsubsection{Apperance Requirements}
The IdentiFisher shall have a professional user interface
\subsubsection{Style Requirements}
The Identifisher shall have a simplistic style

\subsection{Usability and Humanity Requirements}
\subsubsection{Ease of Use Requirements}
The IdentiFisher shall be easy to use by people of ages 10 and above
\subsubsection{Personalization and Internationalization Requirements}
The IdentiFisher shall have English language support
\subsubsection{Learning Requirements}
The IdentiFisher shall be easy to learn by people of ages 10 and above
\subsubsection{Understandability and Politeness Requirements}
The IdentiFisher shall contain no offensive language
\subsubsection{Accessibility Requirements}

\subsection{Performance Requirements}
\subsubsection{Speed and Latency Requirements}
The IdentiFisher shall respond to user input with no noticeable delay
\subsubsection{safety-Critical Requirements}

\subsubsection{Percision or Accuracy Requirements}
The IdentiFisher shall produce correct identifications most of the time
\subsubsection{Reliability and Availability Requirements}
The IdentiFisher shall be operation at all time
\subsubsection{Robustness or Fault-Tolerance Requirements}
The IdentiFisher shall reliably produce correct identifications
\subsubsection{Capacity Requirements}
\subsubsection{Scalability or Extensiblity Requirements}
\subsubsection{Longevity Requirements}
The IdentiFisher shall remain operational for at least 3 years

\subsection{Operational and Environmental Requirements}
\subsubsection{Expected Physical Environment}
\subsubsection{Requirments with Interfacing with Adjacent Systems}
The IdentiFisher shall have unique and simplistic User Interface
\subsubsection{Productization Requirements}
\subsubsection{Release Requirements}
The IdentiFisher shall be ready to be released by April 3, 2016

\subsection{Maintainability and Support Requirements}
\subsubsection{Maintainence Requirements}
The IdentiFisher shall be maintained for the time it stays operational
\subsubsection{Supportability Requirements}
The IdentiFisher shall have tech support for its users
\subsubsection{Adaptiblity Requirements}
The IdentiFisher shall have updates to fix bugs or add new features

\subsection{Security Requirements}
\subsubsection{Access Requirements}
\subsubsection{Integrity Requirements}
\subsubsection{Privacy Requirements}
The IdentiFisher shall not invade privacy of users
\subsubsection{Audit Requirements}
\subsubsection{Immunity Requirements}

\subsection{Cultural and Political Requirements}
\subsubsection{Cultural Requirements}
The IdentiFisher shall not be offensive towards any culture
\subsubsection{Political Requirements}
The IdentiFisher shall not have any political significance

\subsection{Legal Requirements}
\subsubsection{Complicance Requirements}
\subsubsection{Standards Requirements}

\end{document}
