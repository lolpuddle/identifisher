\documentclass{article}
\usepackage{graphicx}
\usepackage{float}

\title{Software Specification Requirements: \\ IdentiFisher \\ Group 7}

\author{
McDonald, Christopher\\
\texttt{1312456}
\and
Guo, Tian\\
\texttt{1327833}
\and
Murray, Shandelle\\
\texttt{1303109}
\and
Cheung, Ocean\\
\texttt{1316057}
\and
Taylor, James\\
\texttt{000000}
}
\vfill
\date{\today}

\begin{document}
\maketitle

\newpage
\tableofcontents
Revision 0: This is the first draft written from the authors listed on the Title page.
\newpage
\section{Introduction}

\subsection{Purpose}
The purpose of the SRS is to provide a detailed account of all the expected functions
and requirements of the software system. It will go into detail regarding the system
as a whole, who we expect to use it, and any relevant information one would need
to endorse or build the system. Lastly, we will outline both the functional and
non-functional requirements of the project that are necessary for the system's
success. The intended audience of this document includes any stakeholders that are involved in this
project. This could include, but is not limited to, the investors, developers, managers,
marketers and human resource workers. Every person who is an entity in the aforementioned
list should take an interest in the details outlined hereafter to ensure that every person
has a clear idea of what the software system should do.

\subsection{Scope}
The software system will be named hereafter as IdentiFisher, which is an Android application.
This system will be a utility application for anyone who fishes, either recreationally or
competivitely. It will service beginner to experienced fishers. Identifisher will allow
the user to give information about a recently caught fish and help to identify what type
of fish it is. From there, it can collect data and track which types of fish are caught where. The aim is to
build a global logging system that will provide percentage catch rates by lake,
educate young, novice fishers, and integrate technology into a relatively non-technological field.


\subsection{Definitions, Acronyms, and Abbreviations}

\textbf{Definition:}

\begin{itemize}
	\item IdentiFisher - The application being referenced in this document.
\end{itemize}
\textbf{Acronyms:}

\begin{itemize}
	\item API - Application Program Interface
	\item OS - Operation System
	\item SDK - Software Development Kit
	\item GPS - Global Positioning System
\end{itemize}

\textbf{Abbreviations:}

\subsection{References}

\subsection{Overview}
Thus far a very brief overview has been provided of the IdentiFisher application, its intended
use, and what we expect a typical user would be. Going forward, we will go into deeper detail
regarding those topics and more. The next section, the overall description, will give far more information regarding
the application and some of the external matters regarding the system. After that, functional
requirements will be listed, with non-functional requirements making up the last section of this document.

\section{Overall Description}

\subsection{Product Perspective}

The IdentiFisher application is similar to other Android applications that,
by user request, analyze textual input or images in order to identify
an entity. It is independent as it is not intended to be used as part
of a larger system; however, it will interface with an online mapping
system in order to perform geolocational functions.

\subsection{Product Functions}
The IdentiFisher application will: 
\begin{itemize}
	\item
	interface with an online mapping system in order to determine the geolocation of the user
	\item
	be able to access a collection of data related to catch rate statistics per body of water
	\item
	determine and display reasonable predictions about the type of fish the user has described
	\item
	allow the user to input textual or pictorial data representing a specific fish
	\item
	request an educated prediction of the type of fish described
	\item
	request statistical information such as catch rate about fish in a specific location
\end{itemize} 


\begin{figure}[H]
	\includegraphics[scale=0.7]{images/contextdiagram}
	\caption{Context Diagram of IdentiFisher Application}
\end{figure}


\subsection{User Characteristics}
\begin{itemize}
	\item
	The IdentiFisher application is intended to be used by beginner to experienced fishers who wish to identify the type of fish they have caught or access geographical catch rate statistics.

	\item
	Technologically, an intended user of the application should have access to a mobile device on which they are capable of installing and accessing an application, establishing an internet connection, as well as generating and inputting textual and pictorial data. 
\end{itemize}

\subsection{Constraints}
\begin{itemize}
	\item
	Since the statistics about a lake are generated by actual users of the application, the method by which data is collected must be subject to strict integrity constraints. The application must employ a method of data verification before adding it to the statistics available to other users of the application. 

	\item
	The application's functionality is constrained by the OS/Software framework through which the application is being written, which is Android/SDK/Java.

	\item
	Although the application does not have any notable budget constraints, there are time constraints to be considered. This document, the Software Requirements Specification, must be complete by February 8, 2016. The application must have been thoroughly tested and be fully functional by April 3th, 2016.
\end{itemize}

\subsection{Assumptions \& Dependencies}
\begin{itemize}
	\item
	An important assumption is that a user of the application is inputting data which represents a real fish and not data that is fake or replicated from another source. This includes inputting images from the internet or simply inputting details about a fish that the user has not actually caught.
	
	\item
	It is assumed that the user has access to a GPS-enabled mobile device with an internet connection through which they may access this application while they are at the location where they have caught the fish they wish to identify.
\end{itemize}

\subsection{Apportioning of Requirements}

\iffalse
There are currently no requirements that will delayed until the future.
\fi

\section{Functional Requirements}
	\textbf{Requirement \#: 1}\\
	\textbf{Description:} The application must accept inputs from the user about the physical specifications of a fish and return an output of what breed of fish the user might have.\\
	\textbf{Rationale: } The application will not be able to give an accurate estimate of what the breed of fish may be without accepting any inputs.

	\noindent\rule{12cm}{0.4pt} \\

	\noindent\textbf{Requirement \#: 2}\\
	\textbf{Description:} The application must be able to access the user's geolocation.\\
	\textbf{Rationale:} The geolocation data could help the user identify what breed of fish the user might have because not all fish exist in every lake.\\

	\noindent\rule{12cm}{0.4pt} \\

	\noindent\textbf{Requirement \#: 3}\\
	\textbf{Description:} The application must display the fishing statistics of different bodies of water by the geolocation specified from the user.\\
	\textbf{Rationale:} The user should be able to access the information about the statistics regarding the odds of catching certain breeds of fish in the specified body of water. \\

	\noindent\rule{12cm}{0.4pt} \\

	\noindent\textbf{Requirement \#: 4}\\
	\textbf{Description:} The application must allow the user to submit their catch results from fishing.\\
	\textbf{Rationale:} The statistics regarding the odds of catching certain breeds of fish in the specified body of water should be accurate and will require constant updates because fish populations could fluctuate over time. \\


\section{Non-Functional Requirements}

\subsection{Look and Feel Requirements}
\subsubsection{Apperance Requirements}
The Identifisher application must have an attractive design and look professional with a 70\% satisfaction rate among users concerning the appearance of the application.
\subsubsection{Style Requirements}
The Identifisher application must have a simple design to promote usability. This is to ensure a small learning curve for users.

\subsection{Usability and Humanity Requirements}
\subsubsection{Ease of Use Requirements}
The IdentiFisher application must be easy to use by people of ages 10 and above.
\subsubsection{Personalization and Internationalization Requirements}
The IdentiFisher application must have English language support.
\subsubsection{Learning Requirements}
The IdentiFisher application must be easy to learn by people of ages 10 and above. Users should take no longer than five minutes in order to familiarize themselves with the application.
\subsubsection{Understandability and Politeness Requirements}
The IdentiFisher application must contain no offensive language.
\subsubsection{Accessibility Requirements}
The IdentiFisher application must be usable by at least 95\% of people with acceptable vision.
\subsection{Performance Requirements}
\subsubsection{Speed and Latency Requirements}
The IdentiFisher application must respond to basic user input within less than 5 seconds.
\subsubsection{Safety-Critical Requirements}
The IdentiFisher application must be safe to use with a 0\% mortality rate.
\subsubsection{Percision or Accuracy Requirements}
The IdentiFisher application must produce correct identifications 75\% of the time provided the input is reasonably dependable.
\subsubsection{Reliability and Availability Requirements}
The IdentiFisher application must be operational and responsive 100\% of the time, except during maintenance or update procedures.
\subsubsection{Robustness or Fault-Tolerance Requirements}
The IdentiFisher application must produce correct identifications 60\% of the time in high-noise environments. This could include
ambigous attributes, low light pictures, and small resolution images.
\subsubsection{Capacity Requirements}
The IdentiFisher application must be able to be used for 1 image or query at a time, for more than 95\% of users.
\subsubsection{Scalability or Extensiblity Requirements}
The IdentFisher application must be designed such that multiple types of inputs can be added with ease.
\subsubsection{Longevity Requirements}
The IdentiFisher application must be available to acquire and remain operational for at least three years.

\subsection{Operational and Environmental Requirements}
\subsubsection{Expected Physical Environment}
The IdentiFisher application will exclusively be running on the Android OS which is on a smartphone.
\subsubsection{Requirments with Interfacing with Adjacent Systems}
The IdentiFisher application must recieve over a 95\% approval rate from the online mapping system API.
\subsubsection{Productization Requirements}
The IdentFisher application must be able to be uploaded to the Android Play Store. It must also be able to be installed from it with no user interaction after the initial prompt.
\subsubsection{Release Requirements}
The IdentiFisher application must be ready to be released by April 3, 2016.

\subsection{Maintainability and Support Requirements}
\subsubsection{Maintainence Requirements}
The IdentiFisher application must be easily maintained for the time it stays operational.
\subsubsection{Supportability Requirements}
The IdentiFisher application must have technical support for all of the users.
\subsubsection{Adaptiblity Requirements}
The IdentiFisher application must adapt to changes and updates in the Android OS and other dependencies.

\subsection{Security Requirements}
\subsubsection{Access Requirements}
The IdentiFisher application will be available to all users which satisfy the Android Play Store requirements to download applications.
\subsubsection{Integrity Requirements}
The IdentiFisher application must encrypt data transferred from the modules of the design to ensure integrity.
\subsubsection{Privacy Requirements}
The IdentiFisher application must adhere to the principle of least privilege and not access information regarding the user that it does not need.
\subsubsection{Audit Requirements}
The IdentiFisher application does not have any Audit Requirements.
\subsubsection{Immunity Requirements}
The IdentiFisher application must store a back-up copy of information regarding which fish were caught in which lake in the event that it is corrupted.

\subsection{Cultural and Political Requirements}
\subsubsection{Cultural Requirements}
The IdentiFisher application must not be offensive towards any culture.
\subsubsection{Political Requirements}
The IdentiFisher application must not have any political affiliation or influence.

\subsection{Legal Requirements}
\subsubsection{Complicance Requirements}
The IdentiFisher application or APIs being used must not go against andy laws or regulations in the country it operates in.
\subsubsection{Standards Requirements}
The IdentiFisher application does not have Standards Requirements.

\newpage
\listoffigures

\end{document}
